%% LyX 2.0.3 created this file.  For more info, see http://www.lyx.org/.
%% Do not edit unless you really know what you are doing.
\documentclass[11pt,letterpaper]{article}
\usepackage[latin9]{inputenc}
\usepackage{textcomp}
\usepackage{amsmath}
\usepackage{amssymb}

\makeatletter

%%%%%%%%%%%%%%%%%%%%%%%%%%%%%% LyX specific LaTeX commands.
\pdfpageheight\paperheight
\pdfpagewidth\paperwidth

\newcommand{\lyxmathsym}[1]{\ifmmode\begingroup\def\b@ld{bold}
  \text{\ifx\math@version\b@ld\bfseries\fi#1}\endgroup\else#1\fi}


%%%%%%%%%%%%%%%%%%%%%%%%%%%%%% User specified LaTeX commands.

\usepackage{fullpage}

\makeatother

\begin{document}
Zackery Field

ID: 23031734

CS 70, Summer 2013

Homework 2

Problem 5 {[}21 Points{]} \bigskip{}

\begin{enumerate}
\item {[}21 Points{]} Application of modular arithmetic.

\begin{enumerate}
\item {[}3 Points{]} Evaluate $(3002+6002*9002)(mod3)$.\\
\begin{eqnarray*}
3002+6002*9002 & mod3\\
2+2*2 & mod3\\
6 & mod3\\
0 & mod3
\end{eqnarray*}

\item {[}3 Points{]} Evaluate $(1002^{3}\lyxmathsym{\textminus}2468*17+4)(mod5)$.\\
\begin{eqnarray*}
1002^{3}\text{\textminus}2468*17+4 & mod5\\
\prod_{i=0}^{n-1}(1002^{2^{i}})^{a_{i}}\text{\textminus}2468*17+4 & mod5\\
(((1002^{2^{1}})^{1})*((1002^{2^{0}})^{1}))\text{\textminus}2468*17+4 & mod5\\
(((1002^{2^{1}})^{1})*((1002^{2^{0}})^{1}))\text{\textminus}2468*17+4 & mod5\\
3-3*2+4 & mod5\\
1 & mod5
\end{eqnarray*}

\item {[}3 Points{]} The numbers $\left\{ 0,1,2,\ldots,19\right\} $are
``representative'' of the congruence classes mod 20. For each of
these classes, determine whether it has an inverse in mod 20, and
if so, state the inverse. If the equation congruence proposition is
left in a form not specifying b, then there exists no inverse.\\
\begin{eqnarray*}
0*b & \equiv & 1(mod20)\\
1*1 & \equiv & 1(mod20)\\
2*b & \equiv & 1(mod20)\\
3*7 & \equiv & 1(mod20)\\
4*b & \equiv & 1(mod20)\\
5*b & \equiv & 1(mod20)\\
6*b & \equiv & 1(mod20)\\
7*3 & \equiv & 1(mod20)\\
8*b & \equiv & 1(mod20)\\
9*9 & \equiv & 1(mod20)\\
10*b & \equiv & 1(mod20)\\
11*11 & \equiv & 1(mod20)\\
12*b & \equiv & 1(mod20)\\
13*17 & \equiv & 1(mod20)\\
14*b & \equiv & 1(mod20)\\
15*b & \equiv & 1(mod20)\\
16*b & \equiv & 1(mod20)\\
17*13 & \equiv & 1(mod20)\\
18*b & \equiv & 1(mod20)\\
19*19 & \equiv & 1(mod20)
\end{eqnarray*}

\item {[}3 Points{]} Evaluate $\frac{5-(19-3)}{7*9}(mod20)$.\\
\begin{eqnarray*}
\frac{5-(19-3)}{7*9} &  & (mod20)\\
\frac{-11}{63} &  & (mod20)\\
-11*(\frac{1}{63})^{-1} & \equiv & 1(mod20)
\end{eqnarray*}
\\
Discovering $(\frac{1}{63})^{-1}$through the extended Euclidean formula:\\
\begin{eqnarray*}
63 & = & 3*20+3\\
20 & = & 6*3+2\\
3 & = & 2*1+1\\
 & Back\\
1 & = & 3-1*2\\
 & = & 3-(20-6*3)\\
 & = & 3-(20+6(63-3(20))\\
 & = & 63-20*3-20+6(63)-18(20)\\
 & = & (7)63-(22)20
\end{eqnarray*}
\\
Continuing from the block abocve:\\
\begin{eqnarray*}
-11*7 & \equiv & 1(mod20)\\
3 & \equiv & (mod20)
\end{eqnarray*}

\item {[}3 Points{]} Use extended Euclidean to find the inverse of 36 mod
55.\\
\begin{eqnarray*}
55 & = & 1*36+19\\
36 & = & 1*19+17\\
19 & = & 1*17+2\\
17 & = & 8*2+1\\
 & Back\\
1 & = & 17-8*2\\
 & = & 17-8*(19-17)\\
 & = & 17-8*(19-(36-19))\\
 & = & (36-19)-8*((55-36)-(36-(55-36)))\\
 & = & (36-(55-36))-8*(55-36-36+55-36)\\
 & = & (2*36-55)-8*(-3*36+2*55)\\
 & = & (26)36-(17)55\\
inverse & = & 26
\end{eqnarray*}

\item {[}3 Points{]} Describe the solutions to $17x\equiv4(mod20)$.\\
\begin{eqnarray*}
17x+20y & = & 4\\
 & Euclid\\
20 & = & 1*17+3\\
17 & = & 5*3+2\\
3 & = & 1*2+1\\
 & Back\\
1 & = & 3-2\\
 & = & 3-(17-5*3)\\
 & = & 3-(17-5*(20-17))\\
 & = & 20-17-17+5*20-5*17\\
 & = & (6)20-(7)17\\
\\
17(-28) & \equiv & 4(mod20)\\
x & = & -28(mod20)\\
 & = & 12
\end{eqnarray*}
\\
$x=12$ is a general solution to $17x\equiv4(mod20)$. The general
solution is described as $12+20k,\: k\in\mathbb{Z}$ 
\item {[}3 Points{]} \end{enumerate}
\end{enumerate}

\end{document}
